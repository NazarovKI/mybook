\subsection{Критерии оценки электронных образовательных ресурсов}
требований.

\begin{longtable}[]{@{}
  >{\raggedright\arraybackslash}p{(\columnwidth - 2\tabcolsep) * \real{0.5502}}
  >{\raggedright\arraybackslash}p{(\columnwidth - 2\tabcolsep) * \real{0.4498}}@{}}
\toprule
\begin{minipage}[b]{\linewidth}\raggedright
\textbf{Дидактические требования.}
\end{minipage} & \begin{minipage}[b]{\linewidth}\raggedright
\textbf{Пути достижения}
\end{minipage} \\
\midrule
\endhead
Научность: опирается на современные методы научного познания:
эксперимент, сравнение, наблюдение, абстрагирование, обобщение,
конкретизация, аналогия, индукция и дедукция, анализ

и синтез. & 1. Брать за основу структуру и подачу материала бумажных
учебников, которые прошли научную экспертизу.

2. Содержание учебников структурируется с учётом современных
исследований в области когнитивных и познавательных процессов. \\
Доступность: соответствие возрастным и индивидуальным особенностям
обучаемых. & 3. Учет рекомендаций возрастной педагогики.
\textbackslash hyperref\{4klass\} \\
Проблемность: использования отдельных компонентов ЭОР вместо
традиционных средств обучения. Пересмотреть пункт & 4. Учёт рекомендаций
учителей, методистов, ученых-педагогов, где и какие проблемы возникают в
выполнении заданий, в усвоении материала. \\
Наглядность: использование аудиовизуальных средств, инфографики,
элементов дополненной реальности. & Использование интерактивных
элементов (перекрестные ссылки, кликабельные рисунки, анимация,
геймификация, аудио и видео-файлы), в т.ч. дополненной реальности. \\
Сознательность: возможность управления ходом событий. Самостоятельность
в работе. & Для электронный учебных изданий важны микротемы, касающиеся
одной проблемы: один вопрос -- одна микротема -- один тест/проверка
знаний, доступные в соответствующем разделе LMS, там же система проверки
и учёта пройденного. \\
Системность и последовательность формирования ЗУН в строго логическом
порядке. &
Ключевая~\textbf{проблема}~\textbf{преподавания}~\textbf{ОРКСЭ}~заключается
в практической неопределенности того, каким должен быть результат этого
безоценочного курса. Все существующие нормативные документы по этому
вопросу содержат такие расплывчатые формулировки, как
«духовно-нравственное воспитание», или же установки на «формирование
личности учащихся, разделяющих российские традиционные духовные
ценности».
\textbackslash url\{https://pravobraz.ru/problemy-prepodavaniya-orkse-i-odnknr-obsudili-v-rgpu-imeni-gercena/\} \\
Единство образовательных, развивающих и воспитательных функций &
Единство реализуется на уровне содержания, оформления и проверочных
заданий, направленных на осознанное освоение учебного материала \\
\textbf{Элементы цифровой дидактики:}

Адаптивность контента к особенностям обучаемых. & Возможность увеличить
шрифт, настроить цвета --- адаптивная верстка, подход mobile first.
(Доступность в первую очередь web-браузере смартфона). \\
Диалог, контроль действий обучаемых, рекомендации по организации обр.
процесса и доступ к справочной информации. & Это задачи Learning
Management System (LMS) \\
формирование стиля мышления, вариативность решений,
информационно-поисковые системы & Постоянное добавление материала,
rolling release. \\
Структурно-функциональная взаимосвязь отдельных компонентов ЭОР. & \\
Непрерывность дид. цикла обучения: выполнение всех звеньев в пределах
одного сеанса работы ЭОР & \\
\bottomrule
\end{longtable}

Таблица 2. Критерии оценки электронных учебников на основе методических
требований.

\begin{longtable}[]{@{}
  >{\raggedright\arraybackslash}p{(\columnwidth - 2\tabcolsep) * \real{0.5542}}
  >{\raggedright\arraybackslash}p{(\columnwidth - 2\tabcolsep) * \real{0.4458}}@{}}
\toprule
\begin{minipage}[b]{\linewidth}\raggedright
\textbf{Методические требования}
\end{minipage} & \begin{minipage}[b]{\linewidth}\raggedright
\textbf{Пути достижения}
\end{minipage} \\
\midrule
\endhead
Предъявление образовательного контента в ЭОР должно строиться с опорой
на взаимосвязь и взаимодействие понятийных, образных и действенных
компонентов мышления. & Разумное соотношение текстовых блоков, рисунков,
схем, таблиц, интерактивных компонентов, аудио и видео материалов. С
учётом особенностей возраста \textbackslash hyperref\{4klass\} \\
\emph{Образовательный контент ЭОР должен отражать систему научных
понятий конкретной учебной дисциплины в виде иерархической структуры,}

\emph{на каждом уровне которой обеспечивается учет как одноуровневых,
так и межуровневых логических взаимосвязей этих понятий.} &
\emph{Древовидное оглавление, раскрывающееся как в проводнике windows,
гиперссылки, категорий, пиктограммы уровней сложности. Система
ограничений доступа к сложным материалам без прохождения простых.} \\
Образовательный контент ЭОР должен предоставлять обучаемому возможность
контролируемых тренировочных действий с целью поэтапного повышения
уровня абстракции и усвоения знаний для осуществления обучаемыми
алгоритмической и эвристической деятельности. & Сложные творческие
задания, которые проверяет учитель. Для выполнения которых требуется
применение полученных в процессе работы с ЭОР знаний, умений и навыков
работы с информацией. ??? \\
\bottomrule
\end{longtable}

Таблица 3. Критерии оценки электронных учебников на основе
психологических требований.

\begin{longtable}[]{@{}
  >{\raggedright\arraybackslash}p{(\columnwidth - 2\tabcolsep) * \real{0.5000}}
  >{\raggedright\arraybackslash}p{(\columnwidth - 2\tabcolsep) * \real{0.5000}}@{}}
\toprule
\begin{minipage}[b]{\linewidth}\raggedright
Психологические требования
\end{minipage} & \begin{minipage}[b]{\linewidth}\raggedright
\end{minipage} \\
\midrule
\endhead
представление образовательного контента в ЭОР должно соответствовать не
только вербально-логическому, но и сенсорно-перцептивному и
представленческому уровням когнитивного процесса. & \\
Представление образовательного контента в ЭОР должно иметь свой тезаурус
и быть ориентировано на лингвистическую композицию конкретного
возрастного контингента обучаемых и специфику его подготовки. &
\textbackslash hyperref\{4klass\} \\
Образовательный контент ЭОР должен быть направлен на развитие

образного и логического мышления. & \\
& \\
& \\
\bottomrule
\end{longtable}

Учет рекомендаций возрастной педагогики.
\textbackslash hyperref\{4klass\}

лингвистическую композицию 4klass

специфику его подготовки 4klass

