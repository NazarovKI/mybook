ЗАКЛЮЧЕНИЕ
В рамках данной работы мы рассмотрели значение ЭОР для образования (раскрыта сущность и особенности ЭОР, классификация, требования, предъявляемые к ЭОР и инструменты для проектирования) и \hyperref[goal]{Среди} конструкторов для создания ЭОР по ОРКСЭ существуют платные платформы LMS с разным уровнем технической поддержки, существуют и конструкторы, построенные на принципах open source. Работа с последним предпологает более глубокого знания языков програмирования и система совместной работы над проектами и контроля версий github. При разработке применяются те же програмные и инструментальные средства и языки, что и вообще в интернете. Для выбора инструмента мы проанализировали специфику преподаваемой дисциплины, целевую аудиторию, уже созданные информационные ресурсы по смежным дисциплинам, и возможные затраты и поняли, что не имеем достаточных исходных данных для того чтобы сделать окончательные выводы по инструментам. В настоящее время ЭОР должен быть доступен в браузере, независимо от платформы.
На первом этапе нужно говорить о концепции и содержании. Мы выявили нехватку ЭОР которые бы отвечали комплексу требований, предъявляемых к электронным образовательным ресурсам. Технические дисциплины обеспечены ЭОР лучше гуманитарных. У социально-гуманитарных дисциплин особая специфика, в рамках школы ЭОР должен прежде всего отвечать требованиям, предъявляемым к электронным учебникам.