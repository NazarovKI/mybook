\newpage
\subsection{Вывод}
В \hyperref[task1]{результате} исследования мы выяснили, что в сущности электронный образовательный ресурс это совокупность учебных, учебно-методических и/или контрольно-измерительных материалов, представленную в виде определенной информационно-технологической конструкции, удобной для изучения и
использования в процессе обучения\cite{jurkina20}. Главная цель использования ЭОР на уроках --- вывести образовательный процесс на новый уровень за счет применения современных инфомационно-коммуникационных технологий. Грамотно сконструированный ЭОР позволяет повысить интерес к обучению и оживить учебный материал за счёт обобщения и систематизации тематических смысловых блоков и визуализации учебного материала, используемого педагогом на уроке.
К электронным образовательным ресурсам предъявляется ряд требований дидактических, методических, эргономических и так далее, призванных обеспечить соответствие образовательного ресурса соответствующим стандартам и повысить качество обучения.

В процессе создания ЭОР на первом этапе нужно говорить не столько о средствах и инструментах, сколько о концепции и содержании. Для понимания, какими инструментами пользоваться для создания ЭОР по ОРКСЭ, нужно проанализировать специфику преподаваемой дисциплины, аудиторию, дидактические и методологические трудности, которые возникают у преподавателей ОРКСЭ. Об этом речь пойдёт во второй главе.